% ----------------------------------------------
%
%	Damian Skrzypiec
% 	02.11.2017	
% 	This script describe process of installation 
% 	cglearn package.
%
% ----------------------------------------------

\lstset{ %
  language=R,                     % the language of the code
  basicstyle=\ttfamily\footnotesize,       % the size of the fonts that are used for the code
  numbers=left,                   % where to put the line-numbers
  numberstyle=\tiny\color{gray},  % the style that is used for the line-numbers
  stepnumber=1,                   % the step between two line-numbers. If it's 1, each line
                                  % will be numbered
  numbersep=5pt,                  % how far the line-numbers are from the code
  backgroundcolor=\color{white},  % choose the background color. You must add \usepackage{color}
  showspaces=false,               % show spaces adding particular underscores
  showstringspaces=false,         % underline spaces within strings
  showtabs=false,                 % show tabs within strings adding particular underscores
  frame=single,                   % adds a frame around the code
  rulecolor=\color{black},        % if not set, the frame-color may be changed on line-breaks within not-black text (e.g. commens (green here))
  tabsize=4,                      % sets default tabsize to 2 spaces
  captionpos=b,                   % sets the caption-position to bottom
  breaklines=true,                % sets automatic line breaking
  breakatwhitespace=false,        % sets if automatic breaks should only happen at whitespace
  title=\lstname,                 % show the filename of files included with \lstinputlisting;
                                  % also try caption instead of title
  keywordstyle=\color{blue},      % keyword style
  commentstyle=\color{dkgreen},   % comment style
  stringstyle=\color{mauve},      % string literal style
  escapeinside={\%*}{*)},         % if you want to add a comment within your code
  morekeywords={*,...}            % if you want to add more keywords to the set
} 

The source code of the implementation of \textit{cglearn} package is placed on GitHub. 
Any R package which is located in GitHub can be installed by using function \textit{install\textunderscore github} from package \textit{devtools}.
The following listing contains R code for installing \textit{cglearn} package.

\begin{lstlisting}[caption={R code for installing \textit{cglearn} package.}]
	if (!require(devtools)) {
		install.packages("devtools")
	}	
	library(devtools)

	devtools::install_github("DSkrzypiec/cglearn")
\end{lstlisting}


The only dependency of \textit{cglearn} package is package \textit{Rcpp} which binds R and C++ language.
The Rcpp package is very common in the world of R packages. Thus such a dependency is not very restrictive. 
Such a short list of dependent packages makes \textit{cglearn} package easy to install, reliable and stable in further development.

