% ----------------------------------------------
%
%	Damian Skrzypiec
% 	03.05.2017
%	Graph Theory Definitions
%
% ----------------------------------------------


This section provides definitions of graph theory objects required for completeness of further sections.
In this section, when is not mention different, $V$ is default notation for set of graph's vertices and 
$E$ is default notation for set of graph's edges. 


% ----------------------
% Undirected edge
% ----------------------
\begin{defi} (Undirected edge) \\
	For vertices $u, v \in V$ we say that there is an undirected edge between vertices $u$ 
	and $v$ if $(u, v) \in E$ and $(v, u) \in E$. Undirected edge between $u$ and $v$ is marked as $u-v$.
\end{defi}


% ----------------------
% Directed edge
% ----------------------
\begin{defi} (Directed edge) \\
	For vertices $u, v \in V$ we say that there is a directed edge from vertex $u$ to vertex $v$ if
	$(u, v) \in E$ and $(v, u) \notin E$. Directed edge from $u$ to $v$ is marked as $u \rightarrow v$.
\end{defi}


% ----------------------
% Parents, Neighbours, Boundry 
% ----------------------
\begin{defi} (Parents, Neighbours, Boundry) \\
	Let $G = (V, E)$ be a graph and $Y \subset V$ be a set of vertices. We define as follows
	\begin{enumerate}
		\item Parents of set $Y$ in graph $G$ is the set defined as 
			$\mbox{Pa}_G(Y) = \left\{ X \ : \ X \rightarrow Z_{Y} \ \mbox{for} \ Z_Y \in Y \right\}$.
		
		\item Neighbors of set $Y$ in graph $G$ is the set defined as 
			$\mbox{Na}_G(Y) = \left\{ X \ : \ X - Z_{Y} \ \mbox{for} \ Z_{Y} \in Y \right\}$.

		\item Boundry of vertex $v \in V$ in graph $G$ is the set defined as $\mbox{Bd}_{G}(v) = \mbox{Pa}_G(v) \cup \mbox{Ne}_G(v)$.
	\end{enumerate}
\end{defi}


% ----------------------
% Sekelton
% ----------------------
\begin{defi} (Skeleton) \\
	Skeleton of graph $G = (V, E)$ is a graph $G' = (V', E')$ where $V = V'$ and the set of edges $E'$
	is obtained by replacing directed edges of set $E$ by undirected edges.
\end{defi}


% ----------------------
% Undirected complete graph
% ----------------------
\begin{defi} (Undirected complete graph) \\ 
	Let $V$ be a set of vertices. Graph $G = (V, E)$ is called undirected complete graph if set of edges $E$ 
	contains undirected edge between any two vertices from $V$.
\end{defi}

% ----------------------
% Route
% ----------------------
\begin{defi} (Route) \\
	A \textit{route} in graph $G = (V, E)$ is a sequence of vertices $(v_0, \dots, v_k)$, $k \ge 0$, such that 
	$$ (v_{i-1}, v_i) \in E \ \  \mbox{or} \ \ (v_i, v_{i-1}) \in E$$
	for $i = 1, \dots, k$. The vertices $v_0$ and $v_k$ are called \textit{terminals}. A route is called descending
	if $(v_{i-1}, v_i) \in E$ for $i = 1, \dots, k$. Descending route from $u$ to $v$ is marked as $u \mapsto v$. 
	A route is called non descending if $(v_i, v_{i-1}) \in E$ or $(v_i, v_{i-1}) \in E \ \wedge \ (v_{i-1}, v_i) \in E$ for 
	$i = 1, \dots, k$. The subset of all vertices $v \in V$ which can be connected by a non descending route with vertex $u$ is
	denoted by $\mbox{An}(u)$.
\end{defi}


% ----------------------
% Path
% ----------------------
\begin{defi} (Path) \\
	A route $r = (v_0, v_1, \dots, v_k)$ in graph $G = (V, E)$ is called a path if all vertices in $r$ are distinct.
\end{defi}


% ----------------------
% Complex
% ----------------------
\begin{defi} \label{complexDef} (Complex) \\
	A path $\pi = (v_1, v_2, \dots, v_k)$ in graph $G = (V, E)$ is called complex if
	\begin{enumerate}
		\item $v_1 \rightarrow v_2$
		\item $\forall_{i \in \left\{ 2, 3, \dots k-2 \right\}} \ v_i - v_{i+1}$
		\item $v_{k-1} \leftarrow v_k$
		\item There is not additional edges in graph $G$ for vertices in path $\pi$.
	\end{enumerate}
	Vertices $v_1$ and $v_k$ are called \textit{parents} of the complex, set of vertices 
	$\left\{v_2, v_3, \dots, v_{k-1} \right\}$ is called \textit{region} of the complex and number
	$k-2$ is the \textit{degree} of the complex.
\end{defi}
Next we define extended version of moral graphs. In Bayesian Networks moral graph is 
an undirected graph obtained from the original graph by adding undirected edges for not connected parents of the same child and then transform all edges into undirected edges. In case of chain graphs there can be situation when there are 
not connected parents of connected children (see \ref{fig:ImmoralCG}). 
This situation can be interpreted as immoral and to be moralized a connection between parents are required.


\begin{ex} (Immorality in chain graph) \\
	\begin{figure}[h]
		\centering
		\vspace{-10pt}
		\begin{tikzpicture}[>=stealth',shorten >=1pt,auto,node distance=2.5cm,
		                    thick,main node/.style={circle,draw,font=\sffamily\Large\bfseries}]
			% Nodes
			  \node[main node] (A) {A};
			  \node[main node] (B) [below of = A] {B};
			  \node[main node] (C) [right of = B] {C};
			  \node[main node] (D) [right of = C] {D};
			  \node[main node] (E) [above of = D] {E};
			
			% Edges
			  \draw[->] (A) -- (B);
			  \draw     (B) -- (C) -- (D);
			  \draw[->] (E) -- (D);
			\end{tikzpicture}
			
		\caption{Immorality in chain graph}			
		\label{fig:ImmoralCG} 
	\end{figure}
\end{ex}

% ----------------------
% Moral Graph
% ----------------------
\begin{defi} \label{moralGraphDef} (Moral Graph) \\
	Let $G = (V, E)$ be a graph. A moral graph $G^{m} = (V, E^{m})$ of graph $G$ is a graph obtained by firstly join parents of complexes in graph $G$ and then replace all edges by undirected edges.
\end{defi}


% ----------------------
% Cycles
% ----------------------
\begin{defi} (Cycle) \\
	A route $r = (v_0, v_1, \dots, v_k)$ in graph $G = (V, E)$ is called a pseudocycle if $v_0 = v_k$ and 
	a cycles if further route is a path and $k \ge 3$.
\end{defi}

A graph with only directed edges is called an \textit{undirected graph}. A graph without directed cycles 
and with only directed edges is called a \textit{directed acyclic graph} (DAG).


% ----------------------
% Chain graph
% ----------------------
\begin{defi}\label{chainGraphDef} (Chain graph)  \\
	A graph $G = (V, E)$ is called a chain graph if it does not have directed (pseudo) cycles.
\end{defi}



% ----------------------
% Section
% ----------------------
\begin{defi} (Section) \\
	A subroute $\sigma = (v_i, \dots, v_j)$ of route $\rho = (v_0, \dots, v_k)$ in graph $G$ is called section if $			\sigma$ is the maximal undirected subroute of route $\rho$. That means $v_i - \dots - v_j$ for $0 \le i \le j 			\le k$. Vertices $v_i$ and $v_j$ are called terminals of section $\sigma$. Further vertex $v_i$ is called a 			head-terminal if $i>0$ and $v_{i-1} \rightarrow v_i$ in graph $G$. Analogically vertex $v_j$ is called 
	a head-terminal if $j<k$ and $v_j \leftarrow v_{j+1}$ in graph $G$.
\end{defi}


A section with two head-terminals is called \textit{collider} section. Otherwise the section is called 
\textit{non collider}. For a given set of vertices $S \subset V$ in graph $G$ and section $\sigma = (v_i, \dots, v_j)$ we say that section is hit by $S$ if $\left\lbrace v_i , \dots, v_j \right\rbrace \cap S \neq \emptyset$. Otherwise we say that section $\sigma$ is outside set $S$.



% ----------------------
% Intervention
% ----------------------
\begin{defi} (Intervention) \\
	A route $\rho$ in graph $G = (V, E)$ is blocked by a subset $S \subset V$ of vertices if and only if there 				exists a section $\sigma$ of route $\rho$ such that one of the following conditions is satisfied.
	
	\begin{enumerate}
		\item Section $\sigma$ is a collider section with respect to $\rho$ and $\sigma$ is outside of $S$.
		\item Section $\sigma$ is non collider section with respect to $\rho$ and $\sigma$ is hit by $S$.
	\end{enumerate}
	
\end{defi}



% ----------------------
%  Intervention lemmma 
% ----------------------

\begin{lemma} \label{lemma1}
	Let $G = (V, E)$ be a chain graph, $\rho$ be a route from vertex $u$ to vertex $v$ in $G$ and $W$ denote set of vertices of route $\rho$.
	If route $\rho$ is blocked by $S \subset V$ in $G$ such that $W \subset S$ then route $\rho$ is blocked by $W$ and any vertex set contaning $W$. 
\end{lemma}

\begin{prf}
	If route $\rho$ is blocked by $S$ then there is a non collider section of $\rho$ which is hit by $S$ or 
	there is a collider section of $\rho$ which is outside of $S$. Any collider section of $\rho$ cannot be outside of $S$ because $W \subset S$.
	With assumption that $W \subset S$ we know that every non collider section is hit by $S$ and therefore is hit also by $W$ and 
	any other vertex set contaning set $W$. 
	\QED
\end{prf}


\begin{ex} (Graph definitions) \\
	Based on the following two graphs (figures \ref{fig:ExampleGraph} and \ref{fig:ExampleMoralGraph}) we 
	present examples of above defined definitions. Let graph 
	presented in figure \ref{fig:ExampleGraph} be denoted as $G$. In graph $G$ as example of descending 
	route is $(A, B, C, D)$ and 
	example of non-descending route is $(D, E, F, G)$. Graph $G$ contains two complexes. Complex $(A, B, C, D, E)$
	is of degree equal to $3$ and the other one $(F, G, H, I)$ is of degree equal to $2$. Graph $G$ contains one cycle
	$(I, J, K, I)$. The Route $(F, G, H, I)$ in graph $G$ contains section $(G, H)$ which is head-to-head section.

	\begin{figure}[h]
		\centering
		\vspace{-10pt}
		\begin{tikzpicture}[>=stealth',shorten >=1pt,auto,node distance=2.5cm,
		                    thick,main node/.style={circle,draw,font=\sffamily\Large\bfseries}]
			% Nodes
			  \node[main node] (A) {A};
			  \node[main node] (B) [above of = A] {B};
			  \node[main node] (C) [right of = B] {C};
			  \node[main node] (D) [right of = C] {D};
			  \node[main node] (E) [below of = D] {E};
			  \node[main node] (F) [right of = E] {F};
			  \node[main node] (G) [above of = F] {G};
			  \node[main node] (H) [right of = G] {H};
			  \node[main node] (I) [below of = H] {I};
			  \node[main node] (J) [right of = I] {J};
			  \node[main node] (K) [above of = J] {K};
			
			% Edges
			  \draw[->] (A) -- (B);
			  \draw     (B) -- (C) -- (D);
			  \draw[->] (E) -- (D);
			  \draw[->] (F) -- (E);
			  \draw[->] (F) -- (G);
			  \draw     (G) -- (H);
			  \draw[->] (I) -- (H);
			  \draw[->] (I) -- (J) -- (K) -- (I);
			\end{tikzpicture}
			
		\caption{Example graph}			
		\label{fig:ExampleGraph} 
	\end{figure}
	
	Graph presented in figure \ref{fig:ExampleMoralGraph} is moral graph of graph $G$. 
	Additional undirected edges $A - E$ and $F - I$ are the result of connecting parents of complexes in 
	the original graph $G$.
	
	\begin{figure}
	 	\centering
	 	\vspace{-10pt}
	 	\begin{tikzpicture}[>=stealth',shorten >=1pt,auto,node distance=2.5cm,
		                    thick,main node/.style={circle,draw,font=\sffamily\Large\bfseries}]
			% Nodes
			  \node[main node] (A) {A};
			  \node[main node] (B) [above of = A] {B};
			  \node[main node] (C) [right of = B] {C};
			  \node[main node] (D) [right of = C] {D};
			  \node[main node] (E) [below of = D] {E};
			  \node[main node] (F) [right of = E] {F};
			  \node[main node] (G) [above of = F] {G};
			  \node[main node] (H) [right of = G] {H};
			  \node[main node] (I) [below of = H] {I};
			  \node[main node] (J) [right of = I] {J};
			  \node[main node] (K) [above of = J] {K};
			
			% Edges
			 \draw (A) -- (B) -- (C) -- (D) -- (E) -- (A);
			 \draw (G) -- (F) -- (E);
			 \draw (F) -- (G) -- (H) -- (I) -- (J) -- (K) -- (I) -- (F);
			\end{tikzpicture}
			
		\caption{Moral graph of graph in figure \ref{fig:ExampleGraph}}			
		\label{fig:ExampleMoralGraph} 
    \end{figure}		
	
	
\end{ex}



