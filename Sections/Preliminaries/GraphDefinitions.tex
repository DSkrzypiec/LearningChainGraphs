% ----------------------------------------------
%
%	Damian Skrzypiec
% 	03.05.2017
%	Graph Theory Definitions
%
% ----------------------------------------------


This section provides definitions of graph theory objects required for completeness of further sections.
In this section, when is not mention different, $V$ is default notation for set of graph's vertices and 
$E$ is default notation for set of graph's edges. 


% ----------------------
% Undirected edge
% ----------------------
\begin{defi}{\textbf{(Undirected edge)}} \\
	For vertices $u, v \in V$ we say that there is an undirected edge between vertices $u$ 
	and $v$ if $(u, v) \in E$ and $(v, u) \in E$. Undirected edge between $u$ and $v$ is marked as $u-v$.
\end{defi}


% ----------------------
% Directed edge
% ----------------------
\begin{defi} {\textbf{(Directed edge)}} \\
	For vertices $u, v \in V$ we say that there is a directed edge from vertex $u$ to vertex $v$ if
	$(u, v) \in E$ and $(v, u) \notin E$. Directed edge from $u$ to $v$ is marked as $u \rightarrow v$.
\end{defi}


% ----------------------
% Sekelton
% ----------------------
\begin{defi} {\textbf{(Skeleton)}} \\
	Skeleton of graph $G = (V, E)$ is a graph $G' = (V', E')$ where $V = V'$ and the set of edges $E'$
	is obtained by replacing directed edges of set $E$ by undirected edges.
\end{defi}


% ----------------------
% Route
% ----------------------
\begin{defi} {\textbf{(Route)}} \\
	A \textit{route} in graph $G = (V, E)$ is a sequence of vertices $(v_0, \dots, v_k)$, $k \ge 0$, such that 
	$$ (v_{i-1}, v_i) \in E \ \  \mbox{or} \ \ (v_i, v_{i-1}) \in E$$
	for $i = 1, \dots, k$. The vertices $v_0$ and $v_k$ are called \textit{terminals}. A route is called descending
	if $(v_{i-1}, v_i) \in E$ for $i = 1, \dots, k$. Descending route from $u$ to $v$ is marked as $u \mapsto v$. 
\end{defi}


% ----------------------
% Path
% ----------------------
\begin{defi} {\textbf{(Path)}} \\
	A route $r = (v_0, v_1, \dots, v_k)$ in graph $G = (V, E)$ is called a path if all vertices in $r$ are distinct.
\end{defi}


% ----------------------
% Complex
% ----------------------
\begin{defi} \label{complexDef} {\textbf{(Complex)}} \\
	A path $\pi = (v_1, v_2, \dots, v_k)$ in graph $G = (V, E)$ is called complex if
	\begin{enumerate}
		\item $v_1 \rightarrow v_2$
		\item $\forall_{i \in \left\{ 2, 3, \dots k-2 \right\}} \ v_i - v_{i+1}$
		\item $v_{k-1} \leftarrow v_k$
		\item There is not additional edges in graph $G$ for vertices in path $\pi$.
	\end{enumerate}
	Vertices $v_1$ and $v_k$ are called \textit{parents} of the complex, set of vertices 
	$\left\{v_2, v_3, \dots, v_{k-1} \right\}$ is called \textit{region} of the complex and number
	$k-2$ is the \textit{degree} of the complex.
\end{defi}


% ----------------------
% Moral Graph
% ----------------------
\begin{defi} \label{moralGraphDef} {\textbf{(Moral Graph)}} \\
	Let $G = (V, E)$ be a graph. A moral graph $G^{m} = (V, E^{m})$ of graph $G$ is a graph obtained by firstly join parents of complexes in graph $G$ and then replace all edges by undirected edges.
\end{defi}


% ----------------------
% Cycles
% ----------------------
\begin{defi} {\textbf{(Cycle)}} \\
	A route $r = (v_0, v_1, \dots, v_k)$ in graph $G = (V, E)$ is called a pseudocycle if $v_0 = v_k$ and 
	a cycles if further route is a path and $k \ge 3$.
\end{defi}

A graph with only directed edges is called an \textit{undirected graph}. A graph without directed cycles 
and with only directed edges is called a \textit{directed acyclic graph} (DAG).


% ----------------------
% Chain graph
% ----------------------
\begin{defi}\label{chainGraphDef} {\textbf{(Chain graph)}}  \\
	A graph $G = (V, E)$ is called a chain graph if it does not have directed (pseudo) cycles.
\end{defi}



% ----------------------
% Section
% ----------------------
\begin{defi} {\textbf{(Section)}} \\
	A subroute $\sigma = (v_i, \dots, v_j)$ of route $\rho = (v_0, \dots, v_k)$ in graph $G$ is called section if $			\sigma$ is the maximal undirected subroute of route $\rho$. That means $v_i - \dots - v_j$ for $0 \le i \le j 			\le k$. Vertices $v_i$ and $v_j$ are called terminals of section $\sigma$. Further vertex $v_i$ is called a 			head-terminal if $i>0$ and $v_{i-1} \rightarrow v_i$ in graph $G$. Analogically vertex $v_j$ is called 
	a head-terminal if $j<k$ and $v_j \leftarrow v_{j+1}$ in graph $G$.
\end{defi}


A section with two head-terminals is called \textit{head-to-head} section. Otherwise the section is called 
\textit{non head-to-head}. For a given set of vertices $S \subset V$ in graph $G$ and section $\sigma = (v_i, \dots, v_j)$ we say that section is hit by $S$ if $\left\lbrace v_i , \dots, v_j \right\rbrace \cap S \neq \emptyset$. Otherwise we say that section $\sigma$ is outside set $S$.



% ----------------------
% Intervention
% ----------------------
\begin{defi} {\textbf{(Intervention)}} \\
	A route $\rho$ in graph $G = (V, E)$ is blocked by a subset $S \subset V$ of vertices if and only if there 				exists a section $\sigma$ of route $\rho$ such that one of the following conditions is satisfied.
	
	\begin{enumerate}
		\item Section $\sigma$ is head-to-head with respect to $\rho$ and $\sigma$ is outside of $S$.
		\item Section $\sigma$ is non head-to-head with respect to $\rho$ and $\sigma$ is hit by $S$.
	\end{enumerate}
	
\end{defi}


\begin{ex}
	Based on the following two graphs we present examples of above defined definitions. Let graph 
	presented in figure \ref{fig:ExampleGraph} be denoted as $G$. In graph $G$ as example of descending route is $(A, B, C, D)$ and 
	example of non-descending route is $(D, E, F, G)$. Graph $G$ contains two complexes. Complex $(A, B, C, D, E)$
	is of degree equal to $3$ and the other one $(F, G, H, I)$ is of degree equal to $2$. Graph $G$ contains one cycle
	$(I, J, K, I)$. Route $(F, G, H, I)$ in graph contains section $(G, H)$ which is head-to-head section.

	\begin{figure}[h]
		\centering
		\vspace{-10pt}
		\begin{tikzpicture}[line cap=round,line join=round,>=triangle 45,x=0.2857142857142857cm,y=0.25cm]
			\clip(2.,2.) rectangle (37.,18.);
			\draw [->,line width=1.2pt] (5.,5.) -- (5.,15.);
			\draw [->,line width=1.2pt] (15.,5.) -- (15.,15.);
			\draw [->,line width=1.2pt] (20.,5.) -- (15.,5.);
			\draw [->,line width=1.2pt] (20.,5.) -- (20.,15.);
			\draw [->,line width=1.2pt] (25.,5.) -- (25.,15.);
			\draw [->,line width=1.2pt] (25.,5.) -- (35.,5.);
			\draw [->,line width=1.2pt] (35.,5.) -- (35.,15.);
			\draw [->,line width=1.2pt] (35.,15.) -- (25.,5.);
			\draw [line width=1.2pt] (5.,15.)-- (10.,15.);
			\draw [line width=1.2pt] (10.,15.)-- (15.,15.);
			\draw [line width=1.2pt] (20.,15.)-- (25.,15.);
			\begin{scriptsize}
				\draw [fill=qqqqff] (5.,5.) circle (2.0pt);
				\draw[color=qqqqff] (5.410110534908851,6.029995001015912) node {$A$};
				\draw [fill=qqqqff] (5.,15.) circle (2.0pt);
				\draw[color=qqqqff] (5.410110534908851,16.01778621683871) node {$B$};
				\draw [fill=qqqqff] (10.,15.) circle (2.0pt);
				\draw[color=qqqqff] (10.434643539187805,16.01778621683871) node {$C$};
				\draw [fill=qqqqff] (15.,15.) circle (2.0pt);
				\draw[color=qqqqff] (15.459176543466759,16.01778621683871) node {$D$};
				\draw [fill=qqqqff] (15.,5.) circle (2.0pt);
				\draw[color=qqqqff] (15.459176543466759,6.029995001015912) node {$E$};
				\draw [fill=qqqqff] (20.,5.) circle (2.0pt);
				\draw[color=qqqqff] (20.422434755010602,6.029995001015912) node {$F$};
				\draw [fill=qqqqff] (20.,15.) circle (2.0pt);
				\draw[color=qqqqff] (20.422434755010602,16.01778621683871) node {$G$};
				\draw [fill=qqqqff] (25.,15.) circle (2.0pt);
				\draw[color=qqqqff] (25.446967759289556,16.01778621683871) node {$H$};
				\draw [fill=qqqqff] (25.,5.) circle (2.0pt);
				\draw[color=qqqqff] (25.446967759289556,6.029995001015912) node {$I$};
				\draw [fill=qqqqff] (35.,5.) circle (2.0pt);
				\draw[color=qqqqff] (35.43475897511236,6.029995001015912) node {$J$};
				\draw [fill=qqqqff] (35.,15.) circle (2.0pt);
				\draw[color=qqqqff] (35.43475897511236,16.01778621683871) node {$K$};
			\end{scriptsize}
			\end{tikzpicture} 
			
	\caption{Example graph}			
	\label{fig:ExampleGraph} 
	\end{figure}
	
	Graph presented in figure \ref{fig:ExampleMoralGraph} is moral graph of graph $G$. 
	Additional edges $[A, E]$ and $[F, I]$ came from connecting parents of two complexes in original graph $G$.
	
	\begin{figure}
	 	\centering
	 	\vspace{-10pt}
	 	\begin{tikzpicture}[line cap=round,line join=round,>=triangle 45,x=0.2777777777777778cm,y=0.2222222222222222cm]
			\clip(2.,2.) rectangle (38.,18.);
			\draw [line width=1.2pt] (5.,15.)-- (10.,15.);
			\draw [line width=1.2pt] (10.,15.)-- (15.,15.);
			\draw [line width=1.2pt] (20.,15.)-- (25.,15.);
			\draw [line width=1.2pt] (5.,15.)-- (5.,5.);
			\draw [line width=1.2pt] (15.,15.)-- (15.,5.);
			\draw [line width=1.2pt] (5.,5.)-- (15.,5.);
			\draw [line width=1.2pt] (15.,5.)-- (20.,5.);
			\draw [line width=1.2pt] (20.,15.)-- (20.,5.);
			\draw [line width=1.2pt] (25.,15.)-- (25.,5.);
			\draw [line width=1.2pt] (20.,5.)-- (25.,5.);
			\draw [line width=1.2pt] (25.,5.)-- (35.,5.);
			\draw [line width=1.2pt] (35.,15.)-- (35.,5.);
			\draw [line width=1.2pt] (35.,15.)-- (25.,5.);
			\begin{scriptsize}
				\draw [fill=qqqqff] (5.,5.) circle (2.0pt);
				\draw[color=qqqqff] (5.559253380426105,6.35141192630792) node {$A$};
				\draw [fill=qqqqff] (5.,15.) circle (2.0pt);
				\draw[color=qqqqff] (5.559253380426105,16.382892069350845) node {$B$};
				\draw [fill=qqqqff] (10.,15.) circle (2.0pt);
				\draw[color=qqqqff] (10.534215077382354,16.382892069350845) node {$C$};
				\draw [fill=qqqqff] (15.,15.) circle (2.0pt);
				\draw[color=qqqqff] (15.590733523469034,16.382892069350845) node {$D$};
				\draw [fill=qqqqff] (15.,5.) circle (2.0pt);
				\draw[color=qqqqff] (15.590733523469034,6.35141192630792) node {$E$};
				\draw [fill=qqqqff] (20.,5.) circle (2.0pt);
				\draw[color=qqqqff] (20.565695220425283,6.35141192630792) node {$F$};
				\draw [fill=qqqqff] (20.,15.) circle (2.0pt);
				\draw[color=qqqqff] (20.565695220425283,16.382892069350845) node {$G$};
				\draw [fill=qqqqff] (25.,15.) circle (2.0pt);
				\draw[color=qqqqff] (25.540656917381533,16.382892069350845) node {$H$};
				\draw [fill=qqqqff] (25.,5.) circle (2.0pt);
				\draw[color=qqqqff] (25.540656917381533,6.35141192630792) node {$I$};
				\draw [fill=qqqqff] (35.,5.) circle (2.0pt);
				\draw[color=qqqqff] (35.57213706042446,6.35141192630792) node {$J$};
				\draw [fill=qqqqff] (35.,15.) circle (2.0pt);
				\draw[color=qqqqff] (35.57213706042446,16.382892069350845) node {$K$};
			\end{scriptsize}
			\end{tikzpicture}
	\caption{Moral graph of graph in figure \ref{fig:ExampleGraph}}			
	\label{fig:ExampleMoralGraph} 
    \end{figure}		
	
	
\end{ex}



