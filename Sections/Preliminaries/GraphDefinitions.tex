% ----------------------------------------------
%
%	Damian Skrzypiec
% 	03.05.2017
%	Graph Theory Definitions
%
% ----------------------------------------------


This section provides definitions of graph theory objects required for completeness of further sections.
In this section, when is not mention different, $V$ is default notation for set of graph's vertices and 
$E$ is default notation for set of graph's edges. 


% ----------------------
% Undirected edge
% ----------------------
\begin{defi}{\textbf{(Undirected edge)}} \\
	For vertices $u, v \in V$ we say that there is an undirected edge between vertices $u$ 
	and $v$ if $(u, v) \in E$ and $(v, u) \in E$. Undirected edge between $u$ and $v$ is marked as $u-v$.
\end{defi}


% ----------------------
% Directed edge
% ----------------------
\begin{defi} {\textbf{(Directed edge)}} \\
	For vertices $u, v \in V$ we say that there is an directed edge from vertice $u$ to vertice $v$ if
	$(u, v) \in E$ and $(v, u) \in E$. Directed edge from $u$ to $v$ is marked as $u \rightarrow v$.
\end{defi}


% ----------------------
% Sekelton
% ----------------------
\begin{defi} {\textbf{(Skeleton)}} \\
	Skeleton of graph $G = (V, E)$ is a graph $G' = (V', E')$ where $V = V'$ and the set of edges $E'$
	is obtained by replacing directed edges of set $E$ by undirected edges.
\end{defi}


% ----------------------
% Route
% ----------------------
\begin{defi} {\textbf{(Route)}} \\
	A \textit{route} in graph $G = (V, E)$ is a sequence of vertices $(v_0, \dots, v_k)$, $k \ge 0$, such that 
	$$ (v_{i-1}, v_i) \in E \ \  \mbox{or} \ \ (v_i, v_{i-1}) \in E$$
	for $i = 1, \dots, k$. The vertices $v_0$ and $v_k$ are called \textit{terminals}. A route is called descending
	if $(v_{i-1}, v_i) \in E$ for $i = 1, \dots, k$. Descending route from $u$ to $v$ is marked as $u \mapsto v$. 
\end{defi}


% ----------------------
% Path
% ----------------------
\begin{defi} {\textbf{(Path)}} \\
	A route $r = (v_0, v_1, \dots, v_k)$ in graph $G = (V, E)$ is called a path if all vertices in $r$ are distinct.
\end{defi}


% ----------------------
% Cycles
% ----------------------
\begin{defi} {\textbf{(Cycle)}} \\
	A route $r = (v_0, v_1, \dots, v_k)$ in graph $G = (V, E)$ is called a pseudocycle if $v_0 = v_k$ and 
	a cycles if further route is a path and $k \ge 3$.
\end{defi}

A graph with only directed edges is called an \textit{undirected graph}. A graph without directed cycles 
and with only directed edges is called a \textit{directed acyclic graph} (DAG).


% ----------------------
% Chain graph
% ----------------------
\begin{defi} {\textbf{(Chain graph)}} \\
	A graph $G = (V, E)$ is called a chain graph if it does not have directed (pseudo) cycles.
\end{defi}



% ----------------------
% Section
% ----------------------
\begin{defi} {\textbf{(Section)}} \\
	A subroute $\sigma = (v_i, \dots, v_j)$ of route $\rho = (v_0, \dots, v_k)$ in graph $G$ is called section if $			\sigma$ is the maximal undirected subroute of route $\rho$. That means $v_i - \dots - v_j$ for $0 \le i \le j 			\le k$. Vertices $v_i$ and $v_j$ are called terminals of section $\sigma$. Further vertex $v_i$ is called a 			head-terminal if $i>0$ and $v_{i-1} \rightarrow v_i$ in graph $G$. Analogically vertex $v_j$ is called 
	a head-terminal if $j<k$ and $v_j \leftarrow v_{j+1}$ in graph $G$.
\end{defi}


A section with two head-terminals is called \textit{head-to-head} section. Otherwise the section is called 
\textit{non head-to-head}. For a given set of vertices $S \subset V$ in graph $G$ and section $\sigma = (v_i, \dots, v_j)$ we say that section is hit by $S$ if $\left\lbrace v_i , \dots, v_j \right\rbrace \cap S \neq \emptyset$. Otherwise we say that section $\sigma$ is outside set $S$.



% ----------------------
% Intervention
% ----------------------
\begin{defi} {\textbf{(Intervention)}} \\
	A route $\rho$ in graph $G = (V, E)$ is blocked by a subset $S \subset V$ of vertices if and only if there 				exists a section $\sigma$ of route $\rho$ such that one of the following conditions is satisfied.
	
	\begin{enumerate}
		\item Section $\sigma$ is head-to-head with respect to $\rho$ and $\sigma$ is outside of $S$.
		\item Section $\sigma$ is non head-to-head with respect to $\rho$ and $\sigma$ is hit by $S$.
	\end{enumerate}
	
\end{defi}



\begin{figure}[h]
	\centering
	\vspace{-10pt}
	\begin{tikzpicture}[line cap=round,line join=round,>=triangle 45,x=1.0cm,y=1.0cm]
		\clip(-2.5,2.5) rectangle (5.,5.5);
		\draw [->] (-2.,5.) -- (2.,5.);
		\draw [->] (2.,5.) -- (4.5,4.);
		\draw [->] (-2.,5.) -- (2.,3.);
		\draw [->] (2.,3.) -- (-2.,3.);
		\draw [->] (2.,3.) -- (4.5,4.);
		\draw [->] (-2.,3.) -- (-2.,5.);
		\begin{scriptsize}
			\draw [fill=uququq] (-2.,5.) circle (2.5pt);
			\draw[color=uququq] (-1.86,5.37) node {$A$};
			\draw [fill=uququq] (2.,5.) circle (2.5pt);
			\draw[color=uququq] (2.14,5.37) node {$B$};
			\draw [fill=uququq] (-2.,3.) circle (2.5pt);
			\draw[color=uququq] (-1.86,3.37) node {$C$};
			\draw [fill=uququq] (2.,3.) circle (2.5pt);
			\draw[color=uququq] (2.14,3.37) node {$D$};
			\draw [fill=uququq] (4.5,4.) circle (2.5pt);
			\draw[color=uququq] (4.64,4.37) node {$E$};
	\end{scriptsize}
	\end{tikzpicture}
	\vspace{-10pt}
	\caption{Example Graph}\label{ExampleGraph} 
\end{figure}


In graph \ref{ExampleGraph} example of descending route is $(A, B, E)$ and example of nondescending route is $(A, D, E, B)$. 
Moreover graph \ref{ExampleGraph} contains a cycle $(A, D, C, A)$.





