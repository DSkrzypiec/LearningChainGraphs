% ----------------------------------------------
%
%	Damian Skrzypiec
% 	04.12.2017	
% 	This script constains results of cglearn package.	
%
% ----------------------------------------------


In this section we present example of usage \textit{cglearn} package.
Listing \ref{exampleRCode} contains R code defining data set (\textit{data.frame} in R) and executing
LCD algorithm implemented in \textit{cglearn} package. In the first part of listing \ref{exampleRCode} (lines 1-8)
we define a mock data set. This approach guarantees easiness of the reproduction of result and furthermore 
in this way we can confront our perception of conditional structure of defined data set with outcome of LCD algorithm.
In lines 10-12 of listing we use function from \textit{cglearn} package to execute LCD algorithm for earlier defined
data set, separation tree build only from one node which contains all of the variables and p-value threshold value set to $0.10$.


\begin{lstlisting}[caption={Example usage of \textit{cglearn} package.}, label={exampleRCode}]
	set.seed(seed = 299938)		
	mockDataFrame <- data.frame(
		A = c(rep(1, 100), rep(0, 100)),
		B = c(rep(1, 80), rep(0, 120)),
		C = c(rep(1, 140), rep(0, 60)),
		D = c(rep(1, 180), rep(0, 20)),
		E = sample(x = c(0, 1), size = 200, replace = TRUE)
	)

	resultIndMatrix <- cglearn::cglearnLCD(data = mockDataFrame,
					       			separationTree = list(1:5),
					       			pValueThr = 0.10)
	resultIndMatrix
\end{lstlisting}

Listing \ref{exampleRResult} presents raw output in R console after execution of listing \ref{exampleRCode}.
The output shows incidence matrix of chain graph obtained by LCD algorithm. Values in the matrix have the following meaning.

\begin{itemize}
	\item Value $2$ represents undirected edge
	\item Value $1$ represents directed edge
	\item Value $0$ represents absence of edge between vertices
\end{itemize}


The outcome of applying LCD algorithm to obtained chain graph seems to be consistent with common sense of dependence structure of 
data set defined in the listing \ref{exampleRCode}.
Chain graph obtained by execution of listing \ref{exampleRCode} is visualized in figure \ref{fig:exampleChainGraph}.

\pagebreak

\begin{lstlisting}[caption={Result of execution listing \ref{exampleRCode}}, label={exampleRResult}]
	  A B C D E
	A 2 2 1 0 0
	B 2 2 0 0 0
	C 0 0 2 2 0
	D 0 0 2 2 0
	E 0 0 0 0 2
\end{lstlisting}

\begin{figure}
	\centering
	\vspace{-10pt}
	\begin{tikzpicture}[>=stealth',shorten >=1pt,auto,node distance=2.5cm,
			    thick,main node/.style={circle,draw,font=\sffamily\Large\bfseries}]

		% Nodes
		\node[main node] (A) at (0, 3) {A};
		\node[main node] (C) at (3, 3) {C};
		\node[main node] (B) at (0, 0) {B};
		\node[main node] (D) at (3, 0) {D};
		\node[main node] (E) at (1.5, -2) {E};

		% Edges
		 \draw[->] (A) -- (C);
		 \draw     (B) -- (A);
		 \draw     (C) -- (D);

	\end{tikzpicture}
		
	\caption{Visualization of result chain graph from listing \ref{exampleRCode}}			
	\label{fig:exampleChainGraph} 
\end{figure}



