%-------------------------------------------------- 
%
%    Damian Skrzypiec
%
%    14.10.2017
%
%    Separation Tree Lemma 
%
%-------------------------------------------------- 

In this section we include lemmas which will be helpful in proving the main theorem for LCD algorithm.
Most lemmas in this section was introduced in original paper about LCD algorithm \cite{CG}.
The following lemma says that every route between two vertices that are separated by some separator in the separation tree
is blocked by this separator. This lemma will be used in the proof of theorem \ref{LCDSkeletonThm}. 

\begin{lemma} \label{lemma9}
	Let $G = (V, E)$ be a chain graph, $\mc{T}(G, \mc{C})$ be a separation tree and $S$ be a separator of $\mc{T}(G, \mc{C})$.
	Suppose that $u \in V_1(S) \setminus S$, $u \in V_2(S) \setminus S$ and $\rho$ is a route from $u$ to $v$ in $G$. Let $W$ denote
	set of vertices of $\rho$. Then $\rho$ is blocked by $W \cap S$ and any vertex set contaning $W \cap S$.
\end{lemma}

\begin{prf}
	We can assume that route $\rho$ is of the form 
	\begin{equation}
		\rho = \left( 
				\overbrace{v^1_1, v^1_2, \dots, v^1_k}^{V_1(S) \setminus S}, 
				\underbrace{s_1, \dots, s_n}_{S}, 
				\overbrace{v^2_1, v^2_2, \dots v^2_m}^{V_2(S) \setminus S}  
			\right) 
	\end{equation}
	Otherwise every vertex of $\rho$ would be in either $V_1(S) \setminus S$ or in $V_2(S) \setminus S$. That would imply existance of
	two adjacent vertices $v \in V_1(S) \setminus S$ and $w \in V_2(S) \setminus S$ and further that would contradictory to condition from separation 
	tree definition $\cSep{V_1(S) \setminus S}{V_2(S) \setminus S}{S}{G}$.
	
	Let $\psi$ be the subroute of the vertices $(v^1_k, s_1, \dots, s_n, v^2_1)$ and $W_{\psi}$ be the vertex set of $\psi$ with vertices $\{v^1_k, v^2_1\}$
	excluded. Since $W_{\psi} \subset S$ there is at least one non collider section of route $\psi$ and, from lemma \ref{lemma1}, every non collider section on $\psi$
	is hit by $W_{\psi}$. Therefore route $\rho$ is blocked by set $S$ or any set contaning $S$ when $\psi$ is part of non collider section of $\rho$.
	To complete proof we have to show that thesis is satisfied in case when subroute $\psi$ is part of collider section of route $\rho$.
	From now on we can assume that route $\rho$ has collider section $\sigma$ of the form
	
	\begin{equation}
		\sigma = v^1_i \rightarrow v^1_{i+1} - \dots - v^1_k - s_1 - \dots - s_n - v^2_1 - \dots v^2_j \leftarrow v^2_{j+1} 
	\end{equation}
	for some $i \in \{1, 2, \dots k \}$ and $j \in \{ 1, 2, \dots, m\}$. We have to consider few corner cases. 
	In case when $v^1_{i} \in S$ then exists a non collider section of $\rho$ which is blocked by $S$ or any set contaning $S$. 
	In case when $v^1_{i} \in V_1(S) \setminus S$ and $v^2_{j+1} \in S$ then exists a non collider section of $\rho$ which is blocked by $S$ or any set contaning $S$.
	In case when $v^2_{j+1} \in V_1(S) \setminus S$ then we can consider a non collider section of route $\rho$ which starts from $v^2_{j+1}$.
	Now it is sufficient to consider regular case when $v^1_{i} \in V_1(S) \setminus S$ and $v^2_{j+1} \in V_2(S) \setminus S$. This case cannot occure
	in separation tree because that would imply $v^1_{i} \not \bigCI v^2_{j+1} \mid S$ which is contradictory to condition from separation tree definition
	$\cSep{V_1(S) \setminus S}{V_2(S) \setminus S}{S}{G}$. 
	\QED
\end{prf}


\begin{lemma} \label{lemma10}
	Let $G = (V, E)$ be a chain graph, $\mc{T}(G, \mc{C})$ be a separation tree of chain graph $G$ and $u, v \in V$ are non adjacent vertices in $G$.
	If $\rho$ is a route from $u$ to $v$ is not contained in $\mbox{An}(u) \cup  \mbox{An}(v)$, then $\rho$ is blocked by any subset of
	$\mbox{An}(u) \cup  \mbox{An}(v)$.
\end{lemma}

\begin{prf}
	Since route $\rho$ is not contained in set $\mbox{An}(u) \cup  \mbox{An}(v)$ there exists vertices $s_1, s_2, t_1, t_2$ such that 
	route $\rho$ is of the form

	\begin{equation}
		\begin{cases}
			\rho = \left(u, \dots, s_1, s_2, \dots, t_1, t_2, \dots, v \right)   \\ 
			\{ s_1, t_2 \} \in \mbox{An}(u) \cup  \mbox{An}(v)  \\ 
			\{ s_2, t_1 \} \cap \left(\mbox{An}(u) \cup  \mbox{An}(v) \right) = \emptyset	
		\end{cases}
	\end{equation}
	In this case we have $s_1 \rightarrow s_2$ and $t_1 \leftarrow t_2$, because otherwise $s_2$ or $t_1$ would be in set 
	$\mbox{An}(u) \cup  \mbox{An}(v)$. Therefore there exists at least one collider section between vertices $s_1$ and $t_2$ 
	on route $\rho$ that is outside of $\mbox{An}(u) \cup  \mbox{An}(v)$. Thus route $\rho$ is blocked by 
	any subset of $\mbox{An}(u) \cup  \mbox{An}(v)$.
	\QED
\end{prf}


\begin{lemma} \label{lemma12}
	Let $G = (V, E)$ be a chain graph, $\mc{T}(G, \mc{C})$ be a separation tree of chain graph $G$
	and $u, v$ are two adjacent vertices. Then there exists node $C \in \mc{T}(G, \mc{C})$ which contains both vertices $u$ and $v$.
\end{lemma}

\begin{prf}
	Let suppose opposite. In this case there exists separator $K$ in $\mc{T}(G), \mc{C}$ such that $u \in V_1(K)$ and $v \in V_2(K)$.
	And by separation tree definition this implies that $\cSep{u}{v}{K}{G}$ and it is not possible becasue $u$ and $v$ are adjacent vertices.
	\QED
\end{prf}



The following lemma \ref{lemma11} states that there exists some set $S$ which c-separates some non adjacent vertices $u$ and $v$
when both of vertices $u$ and $v$ are not contained in the separator connected to their node. 
This lemma will be used in the proof of theorem \ref{LCDSkeletonThm}.


\begin{lemma} \label{lemma11}
	Let $G = (V, E)$ be a chain graph and $\mc{T}(G, \mc{C})$ be a separation tree for $\mc{C} = \{C_1, \dots, C_k \}$. 
	Let $u$ and $v$ be a non adjacent vertices in $G$ such that $u, v \in C_i$ for some $i \in \{1, \dots, k \}$. 
	If $u$ and $v$ are not contained in the same separator connected to $C_i$ then there exists a subset $S$ of $C_i$ such that
	$\cSep{u}{v}{S}{G}$.
\end{lemma}

\begin{prf}
	We show that the following set of vertices 
	\begin{equation} \label{eq:S}
		S = \left( \mbox{An}(u) \cup \mbox{An}(v) \right) \cap \left( C_i \setminus \{ u, v\} \right)
	\end{equation} 
	is the c-separator of vertices $u$ and $v$. Let $\rho$ be a route from $u$ to $v$ in graph $G$. If route $\rho$ is not contained in 
	set $\mbox{An}(u) \cup \mbox{An}(v)$ then $\rho$ is blocked by $S$, by lemma \ref{lemma10}. Therefore from now on we can assume that
	route $\rho$ is contained in set $\mbox{An}(u) \cup \mbox{An}(v)$. To show that $S$ is c-separator we consider the following cases of
	first section of route $\rho$ containing vertex $u$.

	\begin{enumerate}
		\item $u - s_1 - \dots - s_n \leftarrow x$ where $x \neq v$ and $x \in C_i$. \\ 
			In this case $x \in \mbox{An}(u)$ and $x\in C_i$, therefore $x \in S$. That implies
			existance of non collider section containing $x$ which is hit by $S$. Thus $u$ and $v$ 
			are c-separated by $S$.

		\item $u - s_1 - \dots - s_n \rightarrow x - t_1 - \dots - t_m \rightarrow y$ where $x \neq v$ and $x \in C_i$. \\ 
			In this case $x \in \mbox{An}(u) \cup \mbox{An}(v)$ but $x \notin \mbox{An}(u)$, thus $x \in \mbox{An}(v)$.
			Because of $x \in C_i$ and $x \in \mbox{An}(v)$ we know that $x \in S$. Similar to the previous case
			vertices $u$ and $v$ are c-separated because there is at least one non collider section of route $\rho$ 
			containing $x$ which is hit by $S$.

		\item $u - s_1 - \dots - s_n \rightarrow x - t_1 - \dots - t_m \leftarrow y$ where $x \neq v$ and $x,y \in C_i$. \\ 
			In this case $x \not\in \mbox{An}(u)$ and still route $\rho$ is contained in $\mbox{An}(u) \cup \mbox{An}(v)$
			therefore $x \in \mbox{An}(v)$ and thus $y \in \mbox{An}(v)$. That implies $x, y \in S$. Similar to previous 
			cases there is at least one non collider section containing $x$ which is hit by $S$.

		\item $u - s_1 - \dots s_n \rightarrow x - t_1 - \dots - t_m \leftarrow y$ where $x \neq v$, $x \in C_i$ and $y \not\in C_i$. \\ 
			In this case we assume that $y \in C_j$ where $j \neq i$. Next we assume that $y$ belongs to the separator $K$ 
			connected  $C_i$ and next node on the path from $C_i$ to $C_j$ on separation tree $\mc{T}(G, \mc{C})$. To resolve this
			case we divide it into the following cases

			\begin{enumerate}
				\item In case when $\{u, s_1, \dots, s_n \} \cap S \neq \emptyset$ then non collider section $u - s_1 - \dots - s_n$ is hit by $S$.

				\item In case when $\{u, s_1, \dots, s_n \} \cap S = \emptyset$ and $\{x, t_1, \dots, t_k \} \cap S \neq \emptyset$ then 
					collider section $x - t_1 - \dots - t_k$ is outside $S$ and therefore route $\rho$ is blocked.

				\item In case when $\{u, s_1, \dots, s_n \} \cap S = \emptyset$ and $\{x, t_1, \dots, t_k \} \cap S \neq \emptyset$ then
					there is some $t_j \in \{ t_1, \dots, t_k \}$ such that $t_j \neq v$ and $t_j \in C_i \cap \mbox{An}(v)$. 
				 	Since $\{ u, s_1, \dots, s_n \} \cap S = \emptyset$ and $u \bigCI y \mid K$ there should be no 
					collider sections in route $(u, s_1, \dots, s_n, x, t_1, \dots, t_m)$, therefore this case can never happen.
			\end{enumerate}

			Now we consider case when $v \not\in K$. Because of condition $\{ x, t_1, \dots, t_m \} \subset \mbox{An}{v}$ we know that
			there exists at least one arrow '$\rightarrow$' on sub-route of $\rho$ from $y$ to $v$. Futhermore there is no further arrow '$\leftarrow$' 
			closer to ther right end $v$ than it is. Therefore this case identical as one of cases $1.$, $5.$ or $6.$ with $u$ replaced by $v$.

		\item $u - s_1 - \dots - s_m - v$, $u - s_1 \dots - s_m \rightarrow v$ or $u - s_1 - \dots - s_m \leftarrow v$. \\ 
			We know that $s_1 \neq v$ and $s_m \neq u$ because vertices $u$ and $v$ are not adjacent. Suppose that 
			$\{ s_1, \dots, s_m \} \cap S = \emptyset$. That implies, based on definition of set $S$, $\{ s_1, \dots, s_m \} \cap C \subset \{u, v\}$.
			In case when $\{ s_1, \dots, s_m \} \cap S \subset \{ u \}$ then we have non collider section of route $\rho$ which is hit by $S$ and therefore
			it is blocked by $S$. We know that $s_1 \neq u$ and in case when $\{ s_1, \dots, s_m \} \cap C \subset \{ v \}$ then we have $s_1 \not\in C$.
			Suppose that $s_1 \in C_j$ where $C_j$ is a node tree in separation tree $\mc{T}(G, \mc{C})$ which is different from node tree $C_i$ and further
			suppose that $K$ is a separator connected to node $C_i$ on the path between $C_i$ and $C_j$. In this set up $u \in K$, $v \not\in K$ and 
			$s_1 \bigCI v \mid K$.
			However since $\{ s_1, \dots, s_m \} \cap C \subset \{ v \}$ we have $\{ s_1, \dots, s_m \} \cap K = \emptyset$ which is not possible. 
			Therefore we have $\{ s_1, \dots, s_m \} \cap S \neq \emptyset$ and route $\rho$ is blocked.

		\item $u - s_1 - \dots - s_m \rightarrow x$ or $u - s_1 - \dots - s_m \leftarrow x$ where $x \not\in C_i$. \\
			We consider the first case $u - s_1 - \dots - s_m \rightarrow x$. If $\{ u, s_1, \dots, s_m \} \cap S \neq \emptyset$ then 
			route $\rho$ is blocked by $S$ because there is a non collider section of $\rho$ which is hit by $S$. In the opposite case 
			($ \{ u, s_1, \dots, s_m \} \cap S = \emptyset$) suppose that $x \in C_j$ where $C_j$ is a node tree in separation tree $\mc{T}(G, \mc{C})$ 
			which is different from node tree $C_i$ and further suppose that $K$ is a separator connected to node $C_i$ on the path between $C_i$ and $C_j$.
			Now we consider the following two cases.

			\begin{enumerate}
				\item We assume that $v \in K$. This assumption implies $u \not\in K$ and $\{ u, s_1, \dots, s_m \} \cap K \subset \{ v \}$.
					If $\{ u, s_1, \dots, s_m \} \cap K = \{ v \}$ then this case is reduced to the case $5.$. 
					If $\{ u, s_1, \dots, s_m \} \cap K = \emptyset$ then $u \not \bigCI x \mid K$ and that is contradictory to the
					definition of separation tree.

					
				\item We assume that $v \not\in K$. Using similar deduction as in the previous case we have $u \in K$. 
					Since any non collider section of sub-route of route $\rho$ is also non collider section of 
					route $\rho$ we consider sub-route of route $\rho$ starting in $x$. Let $W$ denote set of vertices
					of the sub-route excluding the two end vertices. We know that $W \cap K \subset S \cup \{ u, v \}$.
					In case when $W \cap K \subset S$, by lemma \ref{lemma9}, route $\rho$ is blocked by $S$.
					In case when non collider section of sub-route is hit by $u$ or $v$ we can consider modified sub-route
					starting at that point. Applying the same approach using lemma \ref{lemma9} to the new sub-route 
					we have that $\rho$ is blocked by $S$.

			\end{enumerate}
			Exactly the same approach is used to the other condition in case 6.
			We have considered all the possible cases. Now proof is complete.
			\QED

	\end{enumerate}
	
	
\end{prf}











