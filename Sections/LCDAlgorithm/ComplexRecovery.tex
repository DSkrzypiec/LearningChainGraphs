% ----------------------------------------------
%
%	Damian Skrzypiec
% 	August 2017
%	Complex recovery algorithm (LCD)
%
% ----------------------------------------------

\textbf{Some words}. The following algorithm was introduced in \cite{CG}, chapter 3.3, algorithm 2.

\begin{algorithm}
	\caption{(LCD) Complex Recovery}\label{complexRecoveryAlg}
	
	\textbf{Input:} Perfect conditional independence knowledge about $\mathbb{P}$; the skeleton $G^{'}$ and the set 
					$\mc{S}$ of c-separators obtained in algorithm \ref{skeletonRecoveryAlg}.  \\
	\textbf{Output:} The pattern $G^{*}$ of graph $G$.

	
	\begin{algorithmic}[1]
		\Procedure{ComplexRecovery($ \mc{T}(G, \mc{C})$)}{}
			\State Initialize $G^{*} = G^{'}$ 
	
			\ForAll{$\mbox{ordered pair} \ [u, v] \ : S_{uv} \in \mc{S}$} 
				\ForAll{$ u - w \ \mbox{in} \ G^{*}$}
					\If{$ u \not \bigCI v \mid S_{uv} \cup \{ w \} $}
						\State Orient $u - w$ as $u \rightarrow w$ in $G^{*}$;
					\EndIf
				\EndFor
			\EndFor			
			
			\State \textbf{return:} Pattern of $G^{*}$.
		\EndProcedure
	\end{algorithmic}
\end{algorithm}

By the theorem \ref{LCDComplexProp} and lemma \ref{lemma12} we know that all candidates to edge orientation
are considered in line 3 of the algorithm \ref{complexRecoveryAlg}. 


\begin{ex}
	In this example we present performance of complex recovery algorithm for outcomes from skeleton recovery
	algorithm presented in Example \ref{skeletonRecoveryEx}. If we consider pair $[D, A]$ in outer loop in
	the algorithm we find that $S_{DA} = \{ B \}$ and $D \not \bigCI A \mid \{ B \} \cup \{ C \}$, therefore 
	we orient edge $D \rightarrow C$. Similar we orient edge $A \rightarrow B$, because for pair $[A, D]$ in
	the outer loop we have $S_{AD} = \{ B \}$ and $A \not \bigCI D \mid \{ B \} $. Conditional independence 
	in this case is not satisfied because condition $\cSep{A}{D}{B}{G}$ does not hold and we have assumption
	of faithfulness. For $[D, G]$ in outer loop we have $S_{DG} = \{ F\}$ and 
	$D \not \bigCI G \mid S_{DG} \cup \{ F \}$, hence we orient edge $D \rightarrow F$. Orientation of edge 
	$E \rightarrow F$ is obtained by consideration $[E, H]$ in outer loop and $w = F$. Result of complex
	recovery algorithm is presented in figure \ref{fig:ComplexREcoveryResult}. The edge $[F, H]$ was not oriented
	by the algorithm because condition $F \not\bigCI H \mid F$ is not satisfied. As we mentioned before 
	LCD algorithm creates representative of Markov equivalence class of given chain graph. 
	Chain graphs presented in figure \ref{fig:ComplexREcoveryResult} and \ref{fig:graphForNodeTree} are in the
	same Markov equivalence class.  
\end{ex}


% For Test -- -- --  
% -----------------------------------
%	Viz. of complex recovery alg
% -----------------------------------
\begin{figure}
		\centering
		\vspace{-10pt}
		\begin{tikzpicture}[>=stealth',shorten >=1pt,auto,node distance=2.5cm,
		                    thick,main node/.style={circle,draw,font=\sffamily\Large\bfseries}]
			% Nodes
			  \node[main node] (A) {A};
			  \node[main node] (B) [above of = A] {B};
			  \node[main node] (C) [right of = B] {C};
			  \node[main node] (D) [below of = C] {D};
			  \node[main node] (E) [below of = D] {E};
			  \node[main node] (F) [right of = D] {F};
			  \node[main node] (G) [above of = F] {G};
			  \node[main node] (H) [below of = F] {H};			
			  \node[main node] (I) [below of = A] {I};			
			
			% Edges
			  \draw     (I) -- (A);
			  \draw[->] (A) -- (B);
			  \draw     (B) -- (C);
			  \draw[->] (D) -- (C);
			  \draw[->] (D) -- (F);
			  \draw	    (D) -- (E);
			  \draw[->] (E) -- (F);
			  \draw     (F) -- (G);
			  \draw     (F) -- (H);
			\end{tikzpicture}
			
		\caption{Result of Complex Recovery Algorithm}			
		\label{fig:ComplexREcoveryResult} 
\end{figure}
