% ----------------------------------------------
%
%	Damian Skrzypiec
% 	August 2017
%	LCD Algorithm - section about separation trees.
%
% ----------------------------------------------


% ----------------------------------
%	Tkiz objects definitions
% ----------------------------------

\tikzset{
    triangle/.style={
        draw,regular polygon,
        regular polygon sides=3,
        minimum size = 2.25cm,
        inner sep = 0pt,
        align = center
    }
}

\tikzstyle{block} = [draw, rectangle, minimum width = 0.75cm, minimum height = 0.75cm]


% ------------------------------------------------



For graph $G = (V, E)$ we call set $\mathcal{C} = \left\{ C_1, \dots, C_k\right\}$  
as node set of graph $G$ if $\mathcal{C}$ is a collection of distinct vertex sets such that 
$\forall i \in \left\{1, 2, \dots, k \right\} C_i \subset V$.


\begin{defi} (Node Tree) \\ 
	Let $G = (V, E)$ be a graph and $\mathcal{C} = \left\{ C_1, \dots, C_k\right\}$ be a node set of graph $G$.
	A node tree is a graph $\mc{T}(G, \mc{C}) = (\mc{C} \cup \mc{S}, E)$, where 
	$\mc{S} = \left\{ C_i \cap C_j \ | \ i, j \in \{1, 2, \dots, k\} \right\}$ is set of so-called separators 
	and $E = \left\{ C_i - C_j \ | \ C_i \cap C_j \neq \emptyset \ \mbox{and} \ i, j \in \{1, 2, \dots, k \} \right\}$ 
	is set of undirected edges.
\end{defi}

We will be using graphical convention for representing node trees proposed by Ma, Xie and Geng in \cite{CG}. Regular nodes (from set $\mc{C}$) in node tree will be displayed as triangles and separators (from set $\mc{S}$) will be displayed as rectangles.


\begin{ex} (Node tree) \\
To illustrate node tree definition let's consider graph presented in figure \ref{fig:graphForNodeTree} and node set 
	$\mc{C} = \left\{ \{A, I\}, \{A, B, C, D\}, \{D, E, F\}, \{D, F, G\}, \{E, F, H \} \right\}$. 

	\begin{figure}
		\centering
		\vspace{-10pt}
		\begin{tikzpicture}[>=stealth',shorten >=1pt,auto,node distance=2.5cm,
		                    thick,main node/.style={circle,draw,font=\sffamily\Large\bfseries}]
			% Nodes
			  \node[main node] (A) {A};
			  \node[main node] (B) [above of = A] {B};
			  \node[main node] (C) [right of = B] {C};
			  \node[main node] (D) [below of = C] {D};
			  \node[main node] (E) [below of = D] {E};
			  \node[main node] (F) [right of = D] {F};
			  \node[main node] (G) [above of = F] {G};
			  \node[main node] (H) [below of = F] {H};			
			  \node[main node] (I) [below of = A] {I};

			% Edges
			  \draw[->] (A) -- (B);
			  \draw     (B) -- (C);
			  \draw[->] (D) -- (C);
			  \draw[->] (D) -- (F);
			  \draw		(D) -- (E);
			  \draw[->] (E) -- (F);
			  \draw		(F) -- (G);
			  \draw[->]	(F) -- (H);
			  \draw	    (I) -- (A);
			\end{tikzpicture}
			
		\caption{Example graph for node tree illustration}			
		\label{fig:graphForNodeTree} 
	\end{figure}
	
	
	
	\begin{figure}[h]
		\centering
		\vspace{-10pt}
		\begin{tikzpicture}[>=stealth',shorten >=1pt,auto,node distance=2.5cm,
		                    thick,main node/.style={circle,draw,font=\sffamily\Large\bfseries}]
		                    
			% Nodes
			 \node[triangle] 	(CD) 	at (0, 0) 	{};
			 \node				(CD_up)	at (0, 0.2)	{A B};
			 \node				(CD_down) at (0, -0.2)	{C D};
			 \node[block] 		(C)		at (-2, -1.5) {A};
			 \node[block] 		(D) 	at (2, -1.5)	{D};
			 \node[triangle] 	(ABC) 	at (-2, -3.5)	{};
			 \node 				(ABC_)	at (-2, -3.5) {A I};
			 \node[triangle]	(DEF)	at (2, -3.5)	{};
			 \node				(DEF_)	at (2, -3.5)	{D E F};
			 \node[block]		(DF)	at (3, -5)	{D F};
			 \node[block]		(EF)	at (1, -5)	{E F};
			 \node[triangle]	(DFG)	at (4, -6.5)		{};
			 \node				(DFG_)	at (4, -6.5)		{D F G};	
			 \node[triangle]	(EFH)	at (0, -6.5)	{};
			 \node 				(EFH_)	at (0, -6.5)	{E F H};	 
			 
			\draw (CD) -- (C);
			\draw (CD) -- (D);
			\draw (C) -- (ABC);
			\draw (D) -- (DEF) -- (DF) -- (DFG);
			\draw (DEF) -- (EF) -- (EFH);
		\end{tikzpicture}
			
		\caption{Node tree based on graph \ref{fig:graphForNodeTree} and $\mc{C}$}			
		\label{fig:nodeTree}
	\end{figure}
	
\end{ex}


Notice that if we remove separator $S$ from node tree $\mc{T}(G, \mc{C})$ (or equivalently remove edge that contain separator $S$) then we got two separate node trees $\mc{T}(G, \mc{C}_1(S))$ and $\mc{T}(G, \mc{C}_2(S))$ where 
$\mc{C}_1(S) \cup \mc{C}_2(S) = \mc{C}$. To simplify notation we use 
$$ V_i(S) = \bigcup_{C \in \mc{C}_i(S)} C$$
as union of nodes from node tree $\mc{T}(G, \mc{C}_i(S))$ where $i \in \{1, 2\}$.

\begin{defi} \label{sepTreeDef} (Separation tree) \\
	For given chain graph $G = (V, E)$ and node set $\mathcal{C}$ we say that node tree $\mathcal{T}(G, \mathcal{C})$
	is a separation tree if 

	\begin{enumerate}
		\item $\bigcup_{C \in \mathcal{C}} C = V$ and
		\item for any separator $S$ in node tree $\mathcal{T}(G, \mathcal{C})$ we have  
		$$ \cSep{V_1(S) \setminus S}{V_2(S) \setminus S}{S}{G} $$
	\end{enumerate}	
	
\end{defi}
Let the node tree displayed in figure \ref{fig:nodeTree} be marked as $\mc{T}(G, \mc{C})$. 
The node tree $\mc{T}(G, \mc{C})$ contains four separators $\{A \}$, $\{D \}$, $\{E, F \}$, and $\{D, F \}$. At this point we know that 
$ \bigcup_{C \in \mc{C}} C = V$. To examine if node tree $\mc{T}(G, \mc{C})$ is a separation tree we have to check if second condition 
from definition \ref{sepTreeDef} are satisfied for every separator in $\mc{T}(G, \mc{C})$. 
If we consider separator $\{A \}$ of node tree $\mc{T}(G, \mc{C})$ then we obtain $V_1(\{A \}) = \{ A, I \}$ and 
$V_2(\{A \}) = \{ A, B, C, D, E, F, G, H\}$. The following condition of c-separation 

\begin{equation}
	\cSep{ \{I \} }{ \{B, C, D, E, F, G, H \} }{ \{ A \} }{G}
\end{equation}

holds, because every route from $V_1(\{A \}) \setminus \{A \}$ to $V_2(\{A \}) \setminus \{A \}$ has only non head-to-head sections 
and for each route 
exist some section which is hit by $A$ in graph $G$. Now if we consider separator $\{ D \}$ we have $V_1(\{D \}) = \{ A, B, C, D, I\}$ 
and $V_2(\{D \}) = \{ D, E, F, G, H\}$. There is no route from $\{A, B, C, I \}$ to $\{ E, F, G, H\}$ which would have
a head-to-head section. Additionally every route from $\{A, B, C, I \}$ to $\{ E, F, G, H\}$ contains section which is hit by $D$.
Therefore $\{D \}$ c-separates 
$\{A, B, C, I \}$ and $\{ E, F, G, H\}$ in graph $G$. Using the same argument we could prove that separators $\{E, F\}$ and 
$\{ D, F\}$ also satisfies second condition in c-separation definition \ref{sepTreeDef}. Thus node tree $\mc{T}(G, \mc{C})$ presented
in figure \ref{fig:nodeTree} is actual a separation tree of chain graph $G$ presented in figure \ref{fig:graphForNodeTree}.



