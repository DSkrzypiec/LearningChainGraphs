% ----------------------------------------------
%
%	Damian Skrzypiec
% 	August 2017
%	LCD Algorithm - section about separation trees.
%
% ----------------------------------------------

\tikzset{
    triangle/.style={
        draw,regular polygon,
        regular polygon sides=3,
        minimum size = 1pt,
        text width = 1cm,
        inner sep = 0pt,
        align = center
    }
}

\tikzstyle{block} = [draw, rectangle, minimum width = 0.75cm, minimum height = 0.75cm]



\begin{defi} {\textbf{(Node Tree)}} \\ 
	Let $G = (V, E)$ be a graph and $\mathcal{C} = \left\{ C_1, \dots, C_k\right\}$ be a collection of distinct 
	vertices sets such that $\forall i \in \left\{1, 2, \dots, k \right\} C_i \subset V$. 
	A node tree is a graph which set of vertices are equal to $\mathcal{C}$ and it contains undirected edges between
	every nodes $C_i$ and $C_j$. Moreover for every edge $e$ between $C_i$ and $C_j$ we add separator 
	$S(C_i, C_j) = C_i \cap C_j$ into set of nodes. Nodes $C_i$ are displayed by triangle and separators are
	displayed by rectangles. Notation for node tree is $\mathcal{T}(G, \mathcal{C})$.
\end{defi}


\textbf{TODO: Example of Node tree}

\begin{defi} {\textbf{(Splitting Node Tree by separator)}} \\ 
	Some definition.
\end{defi}


\begin{defi} {\textbf{(Separation tree)}} \\
	For given chain graph $G = (V, E)$ and node set $\mathcal{C}$ we say that node tree $\mathcal{T}(G, \mathcal{C})$
	is a separation tree if 

	\begin{enumerate}
		\item $\cup_{C \in \mathcal{C}} C = V$ and
		\item for any separator $S = S(C_i, C_j)$ in node tree $\mathcal{T}(G, \mathcal{C})$ with $V_1$ 
		and $V_2$ as subtrees obtained by removing separator $S$ we have  
		$$ \cSep{V_1 \setminus S}{V_2 \setminus S}{S}{G} $$
	\end{enumerate}		
	
\end{defi}


\begin{figure}[h]
	\centering
	\vspace{-10pt}
	\begin{tikzpicture}[>=stealth',shorten >=1pt,auto,node distance=2.5cm,
	                    thick,main node/.style={circle,draw,font=\sffamily\Large\bfseries}]
	                    
		% Nodes
		 \node[block, name = A] {A};
		 \node[block, right of = A] (B) {B};
		 \node[triangle, right of = B] (C) {C D H O};
		 \node[block, right of = C] (D) {E, F, G};
		 
		 \draw[->] (A) -- (B);
		 \draw 		(B) -- (C);
		 \draw[->] (C) -- (D);

	\end{tikzpicture}
		
	\caption{Separation Tree}			
\end{figure}