% ----------------------------------------------
%
%	Damian Skrzypiec
% 	August 2017
%	Mathematical basis for LCD algorithm
%
% ----------------------------------------------

In this subsection we introduce mathematical basis for correctness of LCD algorithm.
Theorem \ref{LCDSkeletonThm} provide charaterization of c-separation condition in the separation trees. 
Based on the theorem \ref{LCDSkeletonThm} skeleton recovery phase of LCD algorithm was constructed. 

\begin{thm} \label{LCDSkeletonThm} (\cite{CG}, chapter 3.1, Theorem 3) \\ 
	Let $\mc{T}(G, \mc{C})$ be a separation tree for chain graph $G$. Then vertices $u$ and $v$ are 
	c-separated by some set $S_{uv} \subset V$ in $G$ $(*)$ if and only if one the following conditions hold:
	
	\begin{enumerate}
		\item Vertices $u$ and $v$ are not contained together in any node $C$ of $\mc{T}(G, \mc{C})$,
		
		\item Vertices $u$ and $v$ are contained together in some node $C$, but for any separator $S$ connected
		to $C$, $\{u, v \} \not \subset S$, and there exists $S_{uv}^{'} \subset C$ such 
		that $\cSep{u}{v}{S_{uv}^{'}}{G}$,
		
		\item Vertices $u$ and $v$ are contained together in some node $C$ and both of them belong to some separator
		connected to $C$, but there is a subset $S_{uv}^{'}$ of either $\bigcup_{u \in C^{'}} C^{'}$ or
		$\bigcup_{v \in C^{'}} C^{'}$ such that $\cSep{u}{v}{S_{uv}^{'}}{G}$.
	\end{enumerate}
\end{thm}

\begin{prf}
	At first we will prove the sufficiency of conditions $(1)$, $(2)$ and $(3)$. \\ 

	$(1) \Rightarrow (*)$. \\
	In this case we assume that vertices $u \in C_i$ and $v \in C_j$ are in different nodes of separation tree $\mc{T}(G, \mc{C})$.
	From lemma \ref{lemma9} we know that every route between vertices $u$ and $v$ is blocked by any separator in separation tree
	$\mc{T}(G, \mc{C})$ between nodes $C_i$ and $C_j$. Therefore there is at least one set which c-separates $u$ and $v$ in $G$.

	Implications $(2) \Rightarrow (*)$ and $(3) \Rightarrow (*)$ are satisfied by definition of c-separation. \\ 

	$(*) \Rightarrow (1) \vee (2) \vee (3) $. \\
	Now we assume that there is some set $S \subset V$ which c-separates vertices $u$ and $v$ in graph $G$.
	We can also assume that vertices $u$ and $v$ are contained in the same node of separation tree. Otherwise
	using lemma \ref{lemma9} condition $(1)$ is satisfied. When $u$ and $v$ are contained in the same node $C \in \mc{C}$ we
	have to consider two cases. The first case is when vertices $u$ and $v$ are not contained in the same separator connected to 
	node $C$. The second case is otherwise. The first case is proven by lemma \ref{lemma11}.
	It is know that in chain graph $G$ for any two distinct vertices $u$ and $v$ we have
	\begin{equation}
		\cSep{u}{v}{\mbox{bd}_{G}(u) \cup \mbox{bd}_{G}(v)}{G}
	\end{equation}
	Therefore in the second case we have $\mbox{bd}_{G}(u) \subset \bigcup_{u \in C} C$ and $\mbox{bd}_{G}(v) \subset \bigcup_{v \in C} C$. That
	implies at least one of conditions $2$ and $3$ holds.
	\QED
\end{prf}

 
The following theorem \ref{LCDComplexProp} helps to narrow down number of possibilities to verify 
in the complex recovery phase of the LCD algorithm. 

\begin{thm} \label{LCDComplexProp} (\cite{CG}, chapter 3.1, Proposition 4) \\
	Let $G$ be a chain graph and $\mc{T}(G, \mc{C})$ be a separation tree of $G$. For any complex $\mc{K}$ in $G$, there
	exists some node tree $C$ of $\mc{T}(G, \mc{C})$ such that $\mc{K} \subset C$.
\end{thm}

\begin{prf} 
	Let suppose opposite. Let $\mc{K} = (u, w_1, \dots, w_n, v)$ be a complex in chain graph $G$ such that 
	for any node $C$ in separation tree $\mc{T}(G, \mc{C})$ vertex $u$ and $v$ are not contained together in $C$.
	Now let suppose that $u \in C_u$ and $v \in C_v$ where $C_u$ and $C_v$ are nodes of the separation tree.


	% Vizualization of path between C_u and C_v in the separation tree.	
	\begin{figure}[h]
		\centering
		\vspace{-10pt}
		\begin{tikzpicture}[>=stealth',shorten >=1pt,auto,node distance=2.5cm,
		                    thick,main node/.style={circle,draw,font=\sffamily\Large\bfseries}]
		                    
			% Nodes
			 \node[triangle] 	(Cu) 	at (0, 0) 	{$C_u$};
			 \node[block] 		(S1)	at (2, 0)	{$S_1$};
			 \node[triangle] 	(C1) 	at (4, 0) 	{$C_1$};
			 \node[block] 		(S2)	at (6, 0)	{$S_2$};
			 \node			(dots)	at (8, 0)	{...};
			 \node[block]		(Sk)	at (10, 0)	{$S_k$};
			 \node[triangle] 	(Cv) 	at (12, 0) 	{$C_v$};
			 
			\draw (Cu) -- (S1) -- (C1) -- (S2) -- (dots) -- (Sk) -- (Cv);
		\end{tikzpicture}
			
		\caption{Path between $C_u$ and $C_v$ in separation tree $\mc{T}$}			
		\label{fig:pathInSepTree}
	\end{figure}
	
	Futhermore we introduce $\rho = \{C_u, S_1, C_1, S_2, \dots, S_k, C_v \}$ as a path between $C_u$ and $C_v$ in
	the separation tree $\mc{T}(G, \mc{C})$. We observe that  $u \not \in S_1$ and $v \not \in S_1$ then we have 
	$\{ w_1, w_2, \dots, w_n \} \cap S_1 \neq \emptyset$ and therefore $u \not \bigCI v \mid S_1$. The last implication 
	holds because in this case we have a head-to-head section $u \rightarrow w_1 - w_2 - \dots - w_n \leftarrow v$
	and if $\{ w_1, w_2, \dots, w_n \} \cap S_1 \neq \emptyset$ then this section is not outside of $S_1$. 
	Condition $u \not \bigCI v \mid S_1 $ is contrary to definition of separation tree.
	Because of the above contrary we can assume that $u \in C_1$ and repeate the same process. After finite number
	of iteration we obtain that $u$ and $v$ have to be contained in the same node of the separation tree.
	\QED
\end{prf}


