% ----------------------------------------------
%
%	Damian Skrzypiec
% 	August 2017
%	Mathematical basis for LCD algorithm
%
% ----------------------------------------------

[...] Some introduction.


\begin{thm} (\cite{CG}, chapter 3.1, Theorem 3) \\ 
	Let $\mc{T}(G, \mc{C})$ be a separation tree for chain graph $G$. Then vertices $u$ and $v$ are 
	c-separated by some set $S_{uv} \subset V$ in $G$ if and only if one the following conditions hold:
	
	\begin{enumerate}
		\item Vertices $u$ and $v$ are not contained together in any node $C$ of $\mc{T}(G, \mc{C})$,
		
		\item Vertices $u$ and $v$ are contained together in some node $C$, but for any separator $S$ connected
		to $C$, $\{u, v \} \not \subset S$, and there exists $S_{uv}^{'} \subset C$ such 
		that $\cSep{u}{v}{S_{uv}^{'}}{G}$,
		
		\item Vertices $u$ and $v$ are contained together in some node $C$ and both of them belong to some separator
		connected to $C$, but there is a subset $S_{uv}^{'}$ of either $\bigcup_{u \in C^{'}} C^{'}$ or
		$\bigcup_{v \in C^{'}} C^{'}$ such that $\cSep{u}{v}{S_{uv}^{'}}{G}$.
	\end{enumerate}
\end{thm}


 
\begin{prop} (\cite{CG}, chapter 3.1, Proposition 4) \\
	Let $G$ be a chain graph and $\mc{T}(G, \mc{C})$ be a separation tree of $G$. For any complex $\mc{K}$ in $G$, there
	exists some node tree $C$ of $\mc{T}(G, \mc{C})$ such that $\mc{K} \subset C$.
\end{prop}