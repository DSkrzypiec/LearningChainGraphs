% ----------------------------------------------
%
%	Damian Skrzypiec
% 	August 2017
%	Algorithm Comp. Complexity of LCD alg.
%
% ----------------------------------------------



The skeleton and complex recovery phases are independent in sense of computational complexity, hence we can
find upper bounds for those two phases separately. 
Let suppose that we have unknown chain graph 
$G = (V, E)$ with $|V| = n$ and $|E| = m$. Let further suppose that the input separation tree $\mc{T}$ contains 
tree nodes
$H = \{ C_1, C_2, \dots, C_k \}$. To denote number of elements of separation tree node $C_i$ we use $c_i$ and by
$m$ we denote count of the biggest node in a separation tree $\mc{T}$, that is 
$m = \max \{ c_i \ | \ i \in \{1, 2, \dots, k \} \}$.

The most computational expensive step in the skeleton recovery algorithm is loop in line 5 and verifying 
condition in line 6 of Algorithm \ref{skeletonRecoveryAlg}. For given node in the separation tree $C_i$ looping over
all pairs $\{u, v\} \subset C_i$ is of complexity $\frac{1}{2} c_i \cdot (c_i + 1)$ which is $\mc{O}(c_i^2)$. For given node in the separation
tree and given pair of vertex $\{u, v\}$ verifying condition in line 6 of Algorithm \ref{skeletonRecoveryAlg} is of
cost $2^{c_i}$ because it requires to look over all subsets of $C_i$. Second and third steps of the skeleton
recovery algorithm are less computational complex then the first step. Therefore complexity of the whole algorithm 
is determined by complexity of the first step which can be estimated as follow


\begin{equation} \label{eq:skeletonRecoveryCost}
\begin{split}
	T(\mc{T}) & = \mc{O} \left( \sum_{C_i \in H} \frac{1}{2} c_i (c_i + 1) 2^{c_i} \right)  \le  \\
	&  \le \mc{O} \left( \sum_{C_i \in H} \frac{1}{2}m(m+1)2^{m} \right) =  \\
	& = \mc{O} \left( km^2 2^m \right)
\end{split}
\end{equation}

In the complex recovery phase we have double loop and verifying conditional independence condition.
The first loop is over pairs of vertices $\{ u, v \}$ which are connected, therefore number of iteration
can be bounded by $\mc{O}(m)$. In the second loop for each pair $\{ u, v \}$ there is at most $n$ vertices to check.
Thus computational cost for double loop in the complex recovery algorithm is $\mc{O}(nm)$.
Verifying conditional independence condition in line $5$ of algorithm \ref{complexRecoveryAlg} for
set vertices $u, v$ and $w$ takes $\mc{O}(1)$. The computational complexity of this phase of LCD algorithm
is far more less than the complexity of the skeleton recovery. Therefore computational cost of the LCD algorithm
is dominated by computational cost of the first phase of the algorithm which can be bounded as in inequality \ref{eq:skeletonRecoveryCost}.






