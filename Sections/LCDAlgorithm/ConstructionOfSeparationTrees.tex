% ----------------------------------------------
%
%	Damian Skrzypiec
% 	August 2017
%	Constructing separation trees.
%
% ----------------------------------------------


In this subsection we present method for construction a separation trees by using labeled block ordering.
Concept of labeled block ordering was introduced by Roverto and La Rocca in \cite{LBO}.  
As the name indicates labeled block ordering is a set of blocks of vertices of chain graph 
where the ordering of blocks matters. 


\begin{defi} (labelled block ordering) \\ 
	Let $G = (V, E)$ be a chain graph and $(V_i)_{i = 1}^{n}$ be a partition of a set $V$. A labelled block ordering 
	$\mc{B}$ of chain graph $G$ is a sequence $(V_i^{l_i})_{i = 1}^{n}$ such that $l_i \in \{d, g, u\}$ with convention $V_i = V_i^{g}$.
\end{defi}


\begin{defi} ($\mc{B}$-consistence) \\ 
	Let $G = (V, E)$ be a chain graph and $\mc{B} = (V_i^{l_i})_{i = 1}^{n}$ be a labelled block ordering of $G$. We say that
	$G$ is $\mc{B}$-consistent if 
	
	\begin{enumerate}
		\item every edge connecting vertices $A \in V_i$ and $B \in V_j$ is oriented from $A \rightarrow B$ for $i < j$; 
		\item for every $i$ such that $l_i = u$, the subgraph $G_{V_i}$ is a undirected graph;
		\item for every $i$ such that $l_i = d$, the subgraph $G_{V_i}$ is a DAG;
		\item for every $i$ such that $l_i = g$, the subgraph $G_{V_i}$ may have both directed and undirected edges. 
	\end{enumerate}
\end{defi}

\begin{ex} \label{ExampleLBO}
	Let $V_1 = \{A, B, C, D, R, I \}$, $V_2 = \{G, F\}$ and $V_3 = \{H\}$. Figure \ref{fig:LBOExample} presents a $\mc{B}$-consistent labeled block ordering 
	$\mc{B} = \{ V_1^g, V_2^u, V_3^g\}$ of chain graph from figure \ref{fig:graphForNodeTree}. 
	

	\begin{figure}
		\centering
		\vspace{-10pt}

			
		\begin{tikzpicture}[>=stealth',shorten >=1pt,auto,node distance=2.5cm,
		                    thick,main node/.style={circle,draw,font=\sffamily\Large\bfseries}]

			% Rectangles
			\draw[dotted] (-1, -1) -- (-1, 5) -- (3, 5) -- (3, -1) -- (-1, -1);
			\draw[dotted] (5, 0) -- (5, 4) -- (7, 4) -- (7, 0) -- (5, 0);
			\draw[dotted] (9, 1) -- (9, 3) -- (11, 3) -- (11, 1) -- (9, 1);


			% Nodes
			% First block V_1
			\node[main node] (I) at (0, 0) {I};
			\node[main node] (A) at (0, 2) {A};
			\node[main node] (B) at (0, 4) {B};
			\node[main node] (C) at (2, 4) {C};
			\node[main node] (D) at (2, 2) {D};
			\node[main node] (E) at (2, 0) {E};

			% Second block V_2
			\node[main node] (G) at (6, 3) {G};
			\node[main node] (F) at (6, 1) {F};

			% Third block V_3
			\node[main node] (H) at (10, 2) {H};


			% Edges
			\draw (I) -- (A);
			\draw[->] (A) -- (B);
			\draw (B) -- (C);
			\draw[->] (D) -- (C);
			\draw (D) -- (E);

			\draw[->] (D) -- (F);
			\draw[->] (E) -- (F);
			\draw (G) -- (F);

			\draw[->] (F) -- (H);
		\end{tikzpicture}
		
		\caption{$\mc{B}$-consistent labeled block ordering for chain graph \ref{fig:graphForNodeTree}}
		\label{fig:LBOExample}
	\end{figure}
\end{ex}



Ma, Xie and Geng proposed in \cite{CG} algorithm for creating separation tree based on given labeled block ordering.
Using the main feature of labeled block ordering which is known order of blocks one can construct a DAG of blocks
and then create separation trees from DAG representation of blocks. Figure \ref{fig:FromDAGToSepTree} presents
execution of Algorithm \ref{SepLBOAlg} on labeled block ordering from previous example. On the lefthand side \textit{a)}
of the figure DAG of blocks is presented. It is result of first line of the algorithm. On the righthand side \textit{b)}
of the figure separation tree is presented which is built after second and third line of the algorithm.

\begin{algorithm}
	\caption{(LCD) Separation Tree Construction with Labeled Block Ordering}

	\textbf{Input:} $\mc{B}$-consistent labeled block ordering $\mc{B} = (V_i^{l_i})_{i = 1}^{k}$ of chain graph $G$ \\
	\textbf{Output:} Separation tree $\mc{T}$ of $G$.

	\begin{algorithmic}[1]
			\State Construct a DAG $D$ with blocks $(V_i)_{i = 1}^{k}$
			\State Construct a junction tree $\mc{T}$ by triangulating $D$
			\State Replace each block $V_i$ in $\mc{T}$ by original vertices it contains
			\State \textbf{return:} $\mc{T}$;
	\end{algorithmic}
	\label{SepLBOAlg}
\end{algorithm}



	\begin{figure}
		\centering
		\vspace{-10pt}

			
		\begin{tikzpicture}[>=stealth',shorten >=1pt,auto,node distance=2.5cm,
		                    thick,main node/.style={circle,draw,font=\sffamily\Large\bfseries}]
		
			% DAG
			\node (V1) at (0, 0) {$V_1^g$};
			\node (V2) at (2, 2) {$V_2^u$};
			\node (V3) at (4, 0) {$V_3^g$};

			% Boxes
			\draw (-0.35, -0.35) -- (-0.35, 0.35) -- (0.35, 0.35) -- (0.35, -0.35) -- (-0.35, -0.35);
			\draw (1.65, 1.65) -- (1.65, 2.35) -- (2.35, 2.35) -- (2.35, 1.65) -- (1.65, 1.65);
			\draw (3.65, -0.35) -- (3.65, 0.35) -- (4.35, 0.35) -- (4.35, -0.35) -- (3.65, -0.35);

			% DAG edges
			\draw[->] (V1) -- (V2);
			\draw[->] (V2) -- (V3);

			% Subtitle
			\node (subA) at (2, -2) {a)};
			\node (subB) at (10, -2) {b)};
		


			% Separation tree
			\node (AB) at (7, 1.5) {A B};
			\node (CDE) at (7, 1) {C D E};
			\node (IGF) at (7, 0.5) {I G F};
			
			\node (GF) at (10, 1) {G F};

			\node (GF) at (13, 1.25) {G F};
			\node (H)  at (13, 0.75) {H};


			\draw (5.5, 0) -- (7, 2.5) -- (8.5, 0) -- (5.5, 0);
			\draw (8, 1) -- (9.5, 1);
			\draw (9.5, 0.65) -- (9.5, 1.35) -- (10.5, 1.35) -- (10.5, 0.65) -- (9.5, 0.65);
			\draw (12, 0.25) -- (13, 2.15) -- (14, 0.25) -- (12, 0.25);
			\draw (10.5, 1) -- (12.30, 1);
		\end{tikzpicture}
		\caption{Result of Algorithm \ref{SepLBOAlg} for LBO from \ref{fig:LBOExample}}
		\label{fig:FromDAGToSepTree}
	\end{figure}

