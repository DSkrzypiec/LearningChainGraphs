% ----------------------------------------------
%
%	Damian Skrzypiec
% 	August 2017
%	Skeleton recovery algorithm.
%
% ----------------------------------------------


Some info about SKELETON RECOVERY ALG.

\begin{algorithm}
	\caption{(LCD) Skeleton Recovery}\label{skeletonRecoveryAlg}
	
	\textbf{Input:} A separation tree $ \mc{T}(G, \mc{C})$; 
					perfect conditional independence knowledge about $\mathbb{P}$.  \\
	\textbf{Output:} The skeleton $G^{'}$ of $G$; a set $\mc{S}$ of c-separators.

	
	\begin{algorithmic}[1]
		\Procedure{RecoverySkeleton($ \mc{T}(G, \mc{C})$)}{}
			\State $\mc{S} = \emptyset$
	
			\ForAll{$\mbox{node} \ C_h \in \mc{T}(G, \mc{C})$} 
				\State	Create complete undirected graph $G_h = (C_h, E_h)$;
				\ForAll{$\mbox{vertex pair} \ \{u, v \} \subset C_h $}
					\If{$\exists S_{uv} \subset C_h \ u \bigCI v \mid S_{uv}$}
						\State Delete edge $(u, v)$ from graph $G_h$;
						\State $\mc{S} := \mc{S} \cup S_{uv}$; \Comment{Add set $S_{uv}$ to separators}
					\EndIf
				\EndFor
			\EndFor			
		
			\State Combine all the graphs $(G_h)_{i \in \{1, \dots, H}$ into undirected graph 
			$G^{'} = (V, \bigcup_{h = 1}^{H} E_h)$;
			
			\ForAll{$ \{u, v \} \in G^{'} \ \mbox{contained in more then 
					one node of} \ \mc{T}(G, \mc{C})$}
				\If{$ \exists C_h \ \{u, v \} \subset C_h \ \mbox{and} \ (u, v) \notin E_h$}
					\State Delete the edge $(u, v)$ from $G^{'}$;
				\EndIf
			\EndFor
			
			\ForAll{$ \{u, v \} \in G^{'} \ \mbox{contained in more then 
					one node of} \ \mc{T}(G, \mc{C})$}
					
				\State $N_{uv} := \{S \subset \mbox{ne}_{G^{'}}(u) \cup \mbox{ne}_{G^{'}}(v) \ | \ 
									S \not\subset C_h \ \mbox{and} \ \{u, v\} \subset C_h  \}$
									
				\If{$u \bigCI v \mid S_{uv} \ \mbox{for some} \ S_{uv} \subset N_{uv}$}
					\State Delete edge $(u, v)$ from graph $G^{'}$;
					\State $\mc{S} := \mc{S} \cup S_{uv}$; \Comment{Add set $S_{uv}$ to separators}
				\EndIf
			\EndFor	
			
			\State \textbf{return:} $G^{'}$, $\mc{S}$.
		\EndProcedure
	\end{algorithmic}
\end{algorithm}


% -----------------------------------
%	Example of application LCD alg.
% -----------------------------------

\begin{ex}
	To illustrate execution of the LCD algorithm let's assume we have data which joint distribution is faithful to
	graph presented in figure \ref{fig:graphForNodeTree} but we do not know it's chain graph representation yet.
	Additionally we have separation tree presented on figure \ref{fig:nodeTree}. 
	Result of first phase of skeleton recovery algorithm is presented on figure \ref{fig:LCDfirstPhaseResult}. 
	For every node tree we have local undirected graph representing local
	(in sense of particular node) independence structure. Phase two of skeleton recovery algorithm join local graphs into
	global undirected graph and removes some of redundant edges. 
	Result of execution second phase is represented on figure
	\ref{fig:LCDsecondPhaseResult}. Result of applying skeleton recovery algorithm is the same as outcome from second
	phase of the algorithm, because there isn't pair of random variable satisfying condition from 20th line of Algorithm 
	\ref{skeletonRecoveryAlg}.

\end{ex}

% -----------------------------------
%	Viz. of I phase of LCD alg.
% -----------------------------------
\begin{figure}
	\centering
	\vspace{-10pt}
	\begin{tikzpicture}[>=stealth',shorten >=1pt,auto,node distance=2.5cm,
	                    thick,main node/.style={circle,draw,font=\sffamily\Large\bfseries}]
		% Nodes
		  %Row1
		  \node (A1) at (0, 2) [main node] {A};
		  \node (B1) at (0, 4) [main node] {B};
		  \node (C1) at (2, 4) [main node] {C};
		  
		  \node (C2) at (4, 3) [main node] {C};
		  \node (D2) at (6, 3) [main node] {D};
		  
		  \node (D3) at (8, 4) [main node] {D};
		  \node (E3) at (8, 2) [main node] {E};
		  \node (F3) at (10, 4) [main node] {F};
		  
		  % Row2
		  \node(E4) at (0, -1) [main node] {E};
		  \node(F4) at (2, -1) [main node] {F};
		  \node(H4) at (4, -1) [main node] {H};
		  
		  \node(D5) at (6, -2) [main node] {D};
		  \node(F5) at (8, -2) [main node] {F};
		  \node(G5) at (8, 0) [main node] {G};			
		
		% Edges
		  \draw (A1) -- (B1) -- (C1);
		  \draw (C2) -- (D2);
		  \draw (E3) -- (D3) -- (F3) -- (E3);			  
		  
		  \draw (E4) -- (F4) -- (H4);
		  \draw (D5) -- (F5) -- (G5);
		  
		\end{tikzpicture}
		
	\caption{Result of first phase of LCD algorithm}			
	\label{fig:LCDfirstPhaseResult} 
\end{figure}




% -----------------------------------
%	Viz. of II phase of LCD Alg
% -----------------------------------
\begin{figure}
		\centering
		\vspace{-10pt}
		\begin{tikzpicture}[>=stealth',shorten >=1pt,auto,node distance=2.5cm,
		                    thick,main node/.style={circle,draw,font=\sffamily\Large\bfseries}]
			% Nodes
			  \node[main node] (A) {A};
			  \node[main node] (B) [above of = A] {B};
			  \node[main node] (C) [right of = B] {C};
			  \node[main node] (D) [below of = C] {D};
			  \node[main node] (E) [below of = D] {E};
			  \node[main node] (F) [right of = D] {F};
			  \node[main node] (G) [above of = F] {G};
			  \node[main node] (H) [below of = F] {H};			
			
			% Edges
			  \draw (A) -- (B);
			  \draw (B) -- (C);
			  \draw (C) -- (D);
			  \draw (D) -- (F) -- (E);
			  \draw	(D) -- (E);
			  \draw (F) -- (G);
			  \draw (F) -- (H);
			\end{tikzpicture}
			
		\caption{Result of second phase of LCD algorithm}			
		\label{fig:LCDsecondPhaseResult} 
	\end{figure}